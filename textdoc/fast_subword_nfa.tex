\documentclass[11pt]{jreport}
\usepackage{latexsym}
\usepackage{mathrsfs}
\usepackage{amssymb}
%\usepackage{url}
%\usepackage{lscape}
\usepackage{graphics}
\usepackage{theorem}

\input{A4}
\usepackage{theorem}
\renewcommand{\baselinestretch}{1.25}
\setlength{\parskip}{0.25ex}
\renewcommand{\arraystretch}{0.85}
\begin{document}


\subsection*{定義}

有限アルファベットを $\Sigma$,空文字列を $\varepsilon$ とする.
任意の文字列 $p \in \Sigma^*$ と $t \in \Sigma^*$に対して,
部分文字列パターン $\langle p \rangle$ は,
$s \cdot p \cdot u = t$ となる $s, u \in \Sigma^*$ が存在するとき,
テキスト $t$ に位置 $|s|$ でマッチ(照合)するという.

\begin{defn}[部分語オートマトン subword automata]
ある文字列 $p \in\Sigma^*$ の部分文字列パターン $\langle p \rangle$ がマッチする文字列を受理し,
そうでないものを却下する有限オートマトン $M_p$ を $p$ の部分語オートマトンという.

非決定性の $M_p = (Q, \delta, q_0, F)$ は,以下のように構成できる.
状態の集合を $Q = \{0, \ldots, |p|\}$ で状態数 $|p|+1$, 初期状態を $q_0 = 0$, 受理状態の集合を $F = \{ |p| \}$ とし,
$0 \leq i \leq p, a \in \Sigma$ に対して遷移関係 $\delta \subseteq Q \times \Sigma \times Q$ を以下のように定義する.
\begin{eqnarray*}
\delta(i, a) = \left\{
\begin{array}{ll}
 0 & (i = 0) \\
 1 & (i=0, a = p[0]) \\
 0 & (0 < i < |p|, a \neq p[i]) \\
i+1 & (0 < i < |p|, a = p[i]) \\
|p| & (i = |p|)
\end{array}
\right.
\end{eqnarray*}
\end{defn}

\end{document}

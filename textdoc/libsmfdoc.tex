\documentclass[11pt]{jreport}
\usepackage{latexsym}
\usepackage{mathrsfs}
\usepackage{amssymb}
%\usepackage{url}
%\usepackage{lscape}
\usepackage{graphics}
\usepackage{theorem}

\input{A4}
\usepackage{theorem}
\renewcommand{\baselinestretch}{1.25}
\setlength{\parskip}{0.25ex}
\renewcommand{\arraystretch}{0.85}
\begin{document}

\section{libsmf ドキュメント (20191004版)}

libsmf は,標準MIDIフォーマットファイルを扱うためのライブラリで,
インクルードファイル SMFEvent.h と SMFStream.h のみからり,
使用するプログラムファイルで上記をインクルードすれば使用できる.

\subsection*{SMFEvent.h}
SMF の内容としてあらわれる各 MIDI イベントおよびトラックの先頭マークを表現するデータ構造 SMFEvent と,
その手続き群を定義する.
イベントは,ユーザーが作成することはなく,生成は SMFStream に対する getNextEvent 手続きで取得する際に行われるのみである.

イベントのタイプは 1) システム・エクスクルーシブ,エスケープ形式のシステムエクスクルーシブ,2) メタ,3) MIDI,および 4) ファイルヘッダ,6) トラックヘッダ である.
ノートオンまたはオフ,プログラムチェンジなどのイベントを固定サイズのデータとして扱うことを目的としているため,可変長のテキストデータを持つイベントの場合,そのテキストの先頭8文字分だけを保持する.

構造体として,以下のような直接アクセスできるメンバーを持つ:
\verb+.delta+ デルタタイム.
ノートイベントについては,\verb+.number+ ノートナンバー,
\verb+.velocity+ ベロシティ(0 でノートオフ).
イベントの宛先チャンネルは,\verb+channel()+ で取得する.

個々のイベントに対して,以下の手続きが用意されている.

\subsubsection*{\tt bool isMIDI() const}
MIDI イベントであれば \verb+true+ を返す.

\subsubsection*{\tt bool isNote() const}
ノートオンまたはノートオフイベントであれば \verb+true+ を返す.

\subsubsection*{\tt bool isNoteOff() const}
ノートオフであれば \verb+true+ を返す.

\subsubsection*{\tt bool isNoteOn() const}
ノートオンであれば \verb+true+ を返す.

\subsubsection*{\tt bool isSys() const}
システムエクスクルーシブまたはエスケープ形式のシステムエクスクルーシブイベントであれば \verb+true+ を返す.

\subsubsection*{\tt bool isMeta() const}
メタイベントであれば \verb+true+ を返す.

\subsubsection*{\tt bool isMT() const}
ファイルまたはトラックヘッダであれば \verb+true+ を返す.

\subsubsection*{\tt bool isMTRK() const}
トラックヘッダであれば \verb+true+ を返す.

\subsubsection*{\tt uint8 channel() const}
イベントの宛先チャネルを返す。

\subsubsection*{\tt outstream << オブジェクト}
out stream に SMFEvent オブジェクトを印字出力する.


\subsection*{SMFStream.h}
オープンした SMF バイナリーファイルを SMF イベントのストリームとして扱うための
ラッパー(包みデータ型).

オープンしたバイナリーファイルを引数として宣言し生成する.
メンバー \verb+.smfstream+ が包んでいるバイナリーファイルストリーム.

\subsubsection*{\tt .getNextEvent()}
現在ファイルを読んでいる位置から,イベントを一つ読み取って返す.
読み取り位置は次のイベントの先頭にすすむ.



\end{document}
